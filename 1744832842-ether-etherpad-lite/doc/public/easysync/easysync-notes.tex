\documentclass[12pt]{article}

\usepackage[T1]{fontenc}
\usepackage[USenglish]{babel} 


\begin{document}

\title{Easysync Protocol}
\author{AppJet, Inc., with modifications by the Etherpad Foundation}
\date{\today}

\maketitle

\section{Attributes}

An ``attribute'' is a (key,value) pair such as
\verb|(author,abc123)| or \verb|(bold,true)|.
Sometimes an attribute is treated as an instruction to add
that attribute, in which case an empty value means to
remove it.  So \verb|(bold,)| removes the ``bold''
attribute.  Attributes are interned and given numeric IDs,
so the number ``\verb|6|'' could represent
``\verb|(bold,true)|'', for example.  This mapping is
stored in an attribute pool which may be shared by
multiple changesets.

Entries in the pool must be unique, so that attributes can
be compared by their IDs.  Attribute names cannot contain
commas.

A changeset looks something like the following:

\begin{verbatim}
Z:5g>1|5=2p=v*4*5+1$x
\end{verbatim}

With the corresponding pool containing these entries (among others):

\begin{itemize}
\item[] \verb|4| $\rightarrow$ \verb|(author,1059348573)|
\item[] \verb|5| $\rightarrow$ \verb|(bold,true)|
\end{itemize}

This changeset, together with the attribute pool,
represents inserting a bold letter ``x'' into the middle
of a line.

The string consists of:

\begin{itemize}
\item a letter \verb|Z| (the ``magic character'' and
  format version identifier)
\item a series punctuation marks (operation codes or
  ``opcodes'' for short), together with alphanumerics
  (numeric values in base 36).
\item a dollar sign (\verb|$|)
\item a string of characters used for insertion operations
  (the ``char bank'')
\end{itemize}

In the example above, if we separate out the operations
and convert the numbers to base 10, then we get:
\begin{verbatim}
Z :196 >1 |5=97 =31 *4 *5 +1 $x
\end{verbatim}
Here are descriptions of the operations, where capital
letters are variables:

\begin{description}
\item{{\bf :N}} \quad \\ 
Source text has length $N$ (must be first op)
\item{{\bf >N}} \quad \\
Final text is $N$ (positive) characters longer than source
text (must be second op)
\item{{\bf <N }} \quad \\
Final text is $N$ (positive) characters shorter than
source text (must be second op)
\item{{\bf >0 }} \quad \\
Final text is same length as source text
\item{{\bf +N }} \quad \\
Insert $N$ characters from the bank, none of them newlines
\item{{\bf -N}} \quad \\
Skip over (delete) $N$ characters from the source text,
none of them newlines
\item{{\bf =N}} \quad \\
Keep $N$ characters from the source text, none of them newlines
\item{{\bf |L+N}} \quad \\
Insert $N$ characters from the source text, containing $L$
newlines.  The last character inserted MUST be a newline,
but not the (new) document's final newline.
\item{{\bf |L-N}}  \quad \\
Delete $N$ characters from the source text, containing $L$
newlines. The last character inserted MUST be a newline,
but not the (old) document's final newline.
\item{{\bf |L=N}}  \quad \\
Keep $N$ characters from the source text, containing L
newlines.  The last character kept MUST be a newline, and
the final newline of the document is allowed.
\item{{\bf *I}} \quad \\
Apply attribute $I$ from the pool to the following
\verb|+|, \verb|=|, \verb_|+_, or \verb_|=_ command. In
other words, any number of \verb|*| ops can come before a
\verb_+_, \verb_=_, or \verb_|_ but not between a \verb_|_
and the corresponding \verb_+_ or \verb_=_. If \verb_+_,
text is inserted having this attribute.  If \verb_=_, text
is kept but with the attribute applied as an attribute
addition or removal. Consecutive attributes must be sorted
lexically by (key,value) with key and value taken as
strings.  It's illegal to have duplicate keys for
(key,value) pairs that apply to the same text.  It's
illegal to have an empty value for a key in the case of an
insertion (\verb_+_), the pair should just be omitted.
\end{description}

Characters from the source text that aren't accounted for
are assumed to be kept with the same attributes.

\paragraph{Additional Constraints}

\begin{itemize}
\item Consecutive \verb_+_, \verb_-_, and \verb_=_ ops of
  the same type that could be combined are not allowed.
  Whether combination is possible depends on the
  attributes of the ops and whether each is multiline or
 not.  For example, two multiline deletions can never be
 consecutive, nor can any insertion come after a
 non-multiline insertion with the same attributes.
\item ``No-op'' ops are not allowed, such as deleting 0
  characters.  However, attribute applications that don't
  have any effect are allowed.
\item Characters at the end of the source text cannot be
  explicitly kept with no changes; if the change doesn't
  affect the last $N$ characters, those ``keep'' ops must
  be left off.
\item In any consecutive sequence of insertions (\verb_+_)
  and deletions (\verb_-_) with no keeps (\verb_=_), the
  deletions must come before the insertions.
\item The document text before and after will always end
  with a newline.  This policy avoids a lot of
  special-casing of the end of the document.  If a final
  newline is always added when importing text and removed
  when exporting text, then the changeset representation
  can be used to process text files that may or may not
  have a final newline.
\end{itemize}

\paragraph{Attribution string}

An \emph{attribution string} is a series of inserts with
no deletions or keeps.  For example, ``\verb_*3+8|1+5_''
describes the attributes of a string of length 13, where
the first 8 chars have attribute 3 and the next 5 chars
have no attributes, with the last of these 5 chars being a
newline.  Constraints apply similar to those affecting
changesets, but the restriction about the final newline of
the new document being added doesn't apply.

Attributes in an attribution string cannot be empty, like
``\verb|(bold,)|'', they should instead be absent.


\section{Further Considerations}

\begin{itemize}
\item composing changesets/attributions with different
  pools.
\item generalizing ``applyToAttribution'' to make
  ``mutateAttributionLines'' and ``compose''
\end{itemize}

\section{Using Unicode?}

\begin{itemize}
\item no unicode (for efficient escaping, sightliness)
\item efficient operations for ACE and collab (attributed text, etc.)
\item good for time-slider
\item good for API
\item line-ending aware
X more coherent (deleting or styling text merging with insertion)
\item server-side syntax highlighting?
\item unify author map with attribute pool
\item unify attributed text with changeset rep
\item not: reversible
\item force final newline of document to be preserved
\end{itemize}

\paragraph{Unicode bad!}

\begin{itemize}
\item ugly (hard to read)
\item more complex to parse
\item harder to store and transmit correctly
\item doesn't save all that much space anyway
\item blows up in size when string-escaped
\item embarrassing for API
\end{itemize}


\end{document}
